\documentclass[brazil,ruledheader]{abnt}
\usepackage[T1]{fontenc}
\usepackage[latin1]{inputenc}

\usepackage{palatino}
\usepackage {graphicx}
\usepackage{babel}
\usepackage{setspace}
\usepackage{hyperref}
\usepackage {fancyvrb}
\usepackage{color}

\makeatletter
\makeatother

\definecolor{light-gray}{gray}{0.6}

\begin{document}
\autor{Fausto Guzzo da Costa \ \ \ \ \ \ \ \ \ \ \ \ \ \ \ \ \ \ \ \ \ \ nUsp: 5230736\\ Filipe Del Nero Grillo \ \ \ \ \ \ \ \ \ \ \ \ \ \ \ \ \ \ \ \ \ \ \ nUsp: 5378140 \\ Vinic�us Augusto Tagliatti Zani \ \ \ \ \ \ \ nUsp: 5118935}


\titulo{Avalia��o de Sistemas de Arquivos \\  \normalsize \textcolor{light-gray}{(Ext4, ReiserFS e XFS)}}



\comentario{Monografia apresentada para conclus�o da disciplina de Sistemas Operacionais da p�s-gradua��o do ICMC-USP em 2011}


\instituicao{Instituto de Ci�ncias Matem�ticas e de Computa��o \par Universidade
de S�o Paulo \par ICMC - USP}


\local{S�o Carlos - SP, Brasil}


\data{20 de junho de 2011}

\capa

\folhaderosto

\begin{folhadeaprovacao}
\ \\\ \\\ \\\ \\\ \\\ \\\ \\\ \\\ \\\ \\\ \\\ \\\ \\\ \\\ \\\ \\\ \\
\begin{flushright}{}``\emph{Por mais humilde que seja, um bom}\\
\emph{trabalho inspira uma sensa��o de vit�ria.}\\
{\small Jack Kemp}\end{flushright}{\small \par}
\end{folhadeaprovacao}

\begin{resumo}
\doublespacing

resumo aqui!

\end{resumo}


\tableofcontents{}

\chapter{Introdu��o}\label{cap:introducao}

	
\section{Contexto e Motiva��o}\label{sec:contexto}

		
\section{Objetivos}\label{sec:objetivos}


\section{Organiza��o do trabalho}\label{sec:organizacao}


\chapter{Metodos e Ferramentas}\label{cap:metodos}

\section{XFS}\label{sec:xfs}

\section{Ext4}\label{sec:ext4}

\section{ReiserFS}\label{sec:reiser}

\section{IOzone}\label{sec:iozone}

\section{Avalia��o de desempenho}\label{sec:avaliacao}

\section{Ambiente}\label{sec: ambiente}


\chapter{Planejamento e execu��o}\label{cap:planejamento}

\section{Vari�veis de resposta, fatores e n�veis}\label{sec:fatores}

\section{Experimentos}\label{sec:experimentos}

\chapter{Resultados Obtidos}\label{cap:resultados}

\section{Etapa 1: }\label{sec:etapa1}

\section{Etapa 2: }\label{sec:etapa2}

\section{Etapa 3: }\label{sec:etapa3}	
	
\chapter{Conclus�es}\label{cap:conclusao}

\section{Etapa 1:}\label{sec:etapa1conc}

\section{Etapa 2 }\label{sec:etapa2conc}

\section{Etapa 3 }\label{sec:etapa3conc}	

\section{Considera��es finais}\label{sec:consideracoes}	
	
	
\bibliographystyle{abnt-num}
\bibliography{monografia}

%  \anexo
%  \chapter{Gloss�rio}
%  Teste de ap�ndice


\end{document}